\chapter{Grundlagen}


\section{DML-Technologie}

Ein DML - Distributed Mode Loudspeaker ist ein Flachlautsprecher, welcher Schall über Schwingungsmodi, welche über einen speziellen Körperschallwandler erzeugt und auf die flache Oberfläche übertragen werden. Diese Schwingungen werden von der Gesamtoberfläche abgestrahlt und erzeugen somit hörbaren Schall. Damit stellen DML oder Flachlautsprecher eine fundamentale Alternative zur konventionellen Kolbenstrahler-Technologie dar. Denn im Gegensatz zu klassischen Lautsprechern, wo die Lautsprechermembran angeregt kolbenartige Bewegungen ausführt, basiert die DML-Technologie auf die gezielte Anregung von Biegewellen in einer Platte.


\subsection{Material-Parameter und deren Einfluss}

Hier kommt eine Tabelle hin

\section{Exciter-Position und deren Einfluss}

Multi-Exciter-Konfiguration


\section{Messtechnik}

Messmikrofon, Labview 

\section{VDI-2221: Systematische Produktentwicklung}

Die VDI-Richtlinie 2221 Entwicklung technischer Produkte und Systeme stellt den deutschen Standard für systematische Produktentwicklung dar \footcite{VDI2221}. Die Richtlinie wurde vom VDI - Verein Deutscher Ingenieure herausgegeben und gilt branchenübergreifend für alle Arten technischer Produkte und Systeme. Die Kernprinzipien der VDI-2221 sind systematisches Vorgehen, iteratives Arbeiten, eine frühe Fehlervermeidung und interdisziplinäre Anwendbarkeit. Die Hauptphasen der Produktentwicklung gliedern sich in sieben zentrale Phasen \footcite[S.~11--15]{VDI2221}:
\begin{enumerate}
    \item Klären der Aufgabenstellung
    \item Ermitteln von Funktionen
    \item Suchen nach Lösungsprinzipien
    \item Gliedern in Module
    \item Gestalten der Module
    \item Gestalten des Gesamtprodukts
    \item Ausarbeiten der Dokumentation
\end{enumerate}
Ein zentrales Prinzip ist das iterative Arbeiten: Die Phasen werden nicht starr sequentiell durchlaufen, sondern durch zyklisches Pendeln zwischen Zielen, Anforderungen und Ergebnissen kontinuierlich optimiert \footcite[S.~10]{VDI2221}. Dies ermöglicht frühe Fehlervermeidung und flexible Anpassung an neue Erkenntnisse \footcite{VDI2221}.
Für diese Arbeit strukturiert die VDI-2221 den Entwicklungsprozess des Flach-Lautsprechers. Die Kombination mit dem PDCA-Zyklus, welcher im folgenden Kapitel erläutert wird, gewährleistet kontinuierliche Verbesserung innerhalb der Phasen \footcite[S.~135]{Pahl2013}.



\section{PDCA-Zyklus (Plan-Do-Check-Act)}

Der PDCA-Zyklus ist ein etabliertes Vorgehensmodell zur systematischen Steuerung und kontinuierlichen Verbesserung von Prozessen. Der Begriff PDCA steht für die vier aufeinanderfolgenden Phasen Plan, Do, Check und Act, die gemeinsam einen geschlossenen Regelkreis bilden. Ziel dieses Modells ist es, Prozesse nicht einmalig zu optimieren, sondern sie durch wiederholte Anwendung des Zyklus schrittweise weiterzuentwickeln und dauerhaft zu stabilisieren. Die PDCA-Methode wird insbesondere im Qualitätsmanagement, im Projektmanagement sowie in technischen Entwicklungsprozessen eingesetzt und dient dort als strukturierte Grundlage für methodisches Arbeiten \footcite{Durst2018}.

\begin{figure}[htbp]
    \centering
    \includegraphics[width=0.8\textwidth]{pictures/PDCA}
    \caption{PDCA-Zyklus \footcite{Durst2018}}
    \label{fig:pdca}
\end{figure}

In Abbildung 2.1 ist der PDCA-Zyklus bildlich dargestellt, hierbei sind die vier Phasen klar zu erkennen, welche im Folgenden detailliert beschrieben werden. 
Beginnend mit der Plan-Phase, erfolgt zunächst eine Analyse des bestehenden Ist-Zustands des betrachteten Prozesses. Auf Basis dieser Analyse werden Schwachstellen, Abweichungen oder Verbesserungspotenziale identifiziert. Anschließend werden konkrete Ziele definiert, die innerhalb eines festgelegten Zeitraums erreicht werden sollen. Darauf aufbauend werden Maßnahmen geplant, mit denen diese Ziele umgesetzt werden können. Eine präzise Zieldefinition sowie eine realistische Planung sind in dieser Phase von zentraler Bedeutung, da sie die Grundlage für alle nachfolgenden Schritte darstellen und maßgeblich über den Erfolg des gesamten Zyklus entscheiden \footcite{Durst2018}.

Die Do-Phase umfasst die praktische Umsetzung der in der Plan-Phase definierten Maßnahmen. Dabei werden die geplanten Prozessänderungen oder Verbesserungen realisiert, häufig zunächst in einem begrenzten Umfang oder als Pilotanwendung. Ziel dieser Phase ist es, praktische Erfahrungen zu sammeln und erste Ergebnisse zu erzeugen, ohne den gesamten Prozess sofort vollständig umzustellen. Parallel zur Umsetzung werden relevante Daten erhoben, die später zur Bewertung der Maßnahmen herangezogen werden können \footcite{Durst2018}.

In der Check-Phase erfolgt die systematische Überprüfung der in der Do-Phase erzielten Ergebnisse. Hierbei werden die gemessenen oder beobachteten Resultate mit den zuvor definierten Zielen verglichen. Abweichungen werden analysiert, um festzustellen, ob die umgesetzten Maßnahmen die gewünschte Wirkung erzielt haben. Die Check-Phase dient damit der objektiven Bewertung der Prozessänderungen und stellt sicher, dass Entscheidungen nicht auf subjektiven Eindrücken, sondern auf überprüfbaren Ergebnissen basieren \footcite{Durst2018}.

Die Act-Phase bildet den Übergang von der Bewertung zur erneuten Planung. Auf Grundlage der gewonnenen Erkenntnisse werden erfolgreiche Maßnahmen standardisiert und dauerhaft in den Prozess integriert. Falls die angestrebten Ziele nicht erreicht wurden, werden Korrekturmaßnahmen definiert oder die ursprüngliche Planung angepasst. Mit dem Abschluss der Act-Phase beginnt der PDCA-Zyklus erneut, wodurch ein kontinuierlicher Verbesserungsprozess entsteht. Durch diese iterative Vorgehensweise können Prozesse schrittweise optimiert und langfristig an veränderte Anforderungen angepasst werden \footcite{Durst2018}.