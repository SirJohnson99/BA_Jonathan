\chapter{Grundlagen}


\section{DML-Technologie}


\subsection{Funktionsprinzip und Abstrahlcharakteristik}

\subsection{Material-Parameter und deren Einfluss}


\section{Exciter-Position und deren Einfluss}


\subsection{Multi-Exciter-Konfiguration}


\section{Messtechnik}


\subsection{Frequenzgang-Messung und Interpretation}


\subsection{Bewertung nach DIN 45500}


\section{VDI-2221: Systematische Produktentwicklung}


\section{PDCA-Zyklus (Plan-Do-Check-Act)}

Die PDCA-Methode ist ein etabliertes Vorgehensmodell zur systematischen Steuerung und kontinuierlichen Verbesserung von Prozessen. Der Begriff PDCA steht für die vier aufeinanderfolgenden Phasen Plan, Do, Check und Act, die gemeinsam einen geschlossenen Regelkreis bilden. Ziel dieses Modells ist es, Prozesse nicht einmalig zu optimieren, sondern sie durch wiederholte Anwendung des Zyklus schrittweise weiterzuentwickeln und dauerhaft zu stabilisieren. Die PDCA-Methode wird insbesondere im Qualitätsmanagement, im Projektmanagement sowie in technischen Entwicklungsprozessen eingesetzt und dient dort als strukturierte Grundlage für methodisches Arbeiten (Der Prozessmanager, PDCA-Zyklus).

\begin{figure}[htbp]
    \centering
    \includegraphics[width=0.8\textwidth]{pictures/PDCA}
    \caption{PDCA-Zyklus \cite{QUELLE}}
    \label{fig:pdca}
\end{figure}

In der Plan-Phase erfolgt zunächst eine Analyse des bestehenden Ist-Zustands des betrachteten Prozesses. Auf Basis dieser Analyse werden Schwachstellen, Abweichungen oder Verbesserungspotenziale identifiziert. Anschließend werden konkrete Ziele definiert, die innerhalb eines festgelegten Zeitraums erreicht werden sollen. Darauf aufbauend werden Maßnahmen geplant, mit denen diese Ziele umgesetzt werden können. Eine präzise Zieldefinition sowie eine realistische Planung sind in dieser Phase von zentraler Bedeutung, da sie die Grundlage für alle nachfolgenden Schritte darstellen und maßgeblich über den Erfolg des gesamten Zyklus entscheiden (Der Prozessmanager, PDCA-Zyklus).

Die Do-Phase umfasst die praktische Umsetzung der in der Plan-Phase definierten Maßnahmen. Dabei werden die geplanten Prozessänderungen oder Verbesserungen realisiert, häufig zunächst in einem begrenzten Umfang oder als Pilotanwendung. Ziel dieser Phase ist es, praktische Erfahrungen zu sammeln und erste Ergebnisse zu erzeugen, ohne den gesamten Prozess sofort vollständig umzustellen. Parallel zur Umsetzung werden relevante Daten erhoben, die später zur Bewertung der Maßnahmen herangezogen werden können (Der Prozessmanager, PDCA-Zyklus).

In der Check-Phase erfolgt die systematische Überprüfung der in der Do-Phase erzielten Ergebnisse. Hierbei werden die gemessenen oder beobachteten Resultate mit den zuvor definierten Zielen verglichen. Abweichungen werden analysiert, um festzustellen, ob die umgesetzten Maßnahmen die gewünschte Wirkung erzielt haben. Die Check-Phase dient damit der objektiven Bewertung der Prozessänderungen und stellt sicher, dass Entscheidungen nicht auf subjektiven Eindrücken, sondern auf überprüfbaren Ergebnissen basieren (Der Prozessmanager, PDCA-Zyklus).

Die Act-Phase bildet den Übergang von der Bewertung zur erneuten Planung. Auf Grundlage der gewonnenen Erkenntnisse werden erfolgreiche Maßnahmen standardisiert und dauerhaft in den Prozess integriert. Falls die angestrebten Ziele nicht erreicht wurden, werden Korrekturmaßnahmen definiert oder die ursprüngliche Planung angepasst. Mit dem Abschluss der Act-Phase beginnt der PDCA-Zyklus erneut, wodurch ein kontinuierlicher Verbesserungsprozess entsteht. Durch diese iterative Vorgehensweise können Prozesse schrittweise optimiert und langfristig an veränderte Anforderungen angepasst werden (Der Prozessmanager, PDCA-Zyklus).