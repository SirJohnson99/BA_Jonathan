\chapter{Lösung der Aufgabenstellung}


\section{Konzeptentwicklung}

Die Entwicklung eines DML-Lautsprechersystems erfordert zunächst eine grundsätzliche Entscheidung bezüglich der Systemarchitektur. Im Rahmen dieser Arbeit werden drei fundamentale Systemkonfigurationen hinsichtlich ihrer technischen Eignung, akustischen Performance und Praktikabilität evaluiert: das Fullrange-System ohne Frequenzweiche, das 2-Wege-System mit Subwoofer-Integration sowie das 3-Wege-System mit separater Bass-, Mitten- und Höhenwiedergabe.
Fullrange-System ohne Frequenzweiche

Das Fullrange-Konzept stellt die einfachste Implementierung eines DML-Lautsprechers dar, bei welcher das Panel den gesamten hörbaren Frequenzbereich ohne Frequenzweiche abdeckt. [Bai \& Huang 2001] dokumentieren in ihrer grundlegenden Arbeit zur Entwicklung von Panel-Lautsprechersystemen, dass die untere Grenzfrequenz eines DML-Panels maßgeblich durch dessen Abmessungen determiniert wird. Für ein Panel mit den Dimensionen 60 cm × 45 cm ergibt sich aus ihren experimentellen Daten eine untere Grenzfrequenz im Bereich von 95-100 Hz [Bai \& Huang 2001, S. 2755]. Diese Limitation resultiert aus der Wellenlängenabhängigkeit der Biegewellen: Zur effizienten Abstrahlung tiefer Frequenzen ist eine minimale Panel-Diagonale erforderlich, welche in einem proportionalen Verhältnis zur Wellenlänge steht.
Der fundamentale Vorteil eines Fullrange-Systems liegt in der Vermeidung jeglicher Frequenzweichen, wodurch phasenbedingte Kohärenzverluste sowie Interferenzeffekte eliminiert werden [Harris \& Hawksford 2000]. Dies ist insbesondere für den kritischen Vokalbereich zwischen 500 Hz und 4 kHz von Bedeutung, in welchem die menschliche Sprachverständlichkeit primär determiniert wird [Fastl \& Zwicker 2007, S. 187]. Ein Fullrange-System gewährleistet somit die kohärenteste Schallabstrahlung über das gesamte Spektrum.
Die primäre Limitation dieses Konzepts besteht jedoch in der unzureichenden Bass-Extension. [Bai \& Huang 2001, S. 2758] quantifizieren den Wirkungsgrad von DML-Panels mit lediglich 0,039\%, verglichen mit 0,089\% für konventionelle Lautsprecher – eine Reduktion um den Faktor 2,3. In Kombination mit der limitierten unteren Grenzfrequenz resultiert dies in einer insuffizienten Wiedergabe von bassintensiven Musikgenres. [Engineering Radio Blog 2021] berichtet von einem experimentellen Fullrange-DML mit einer 4'×2' Platte (122 cm × 61 cm), welches trotz der substantiell größeren Abmessungen eine untere Grenzfrequenz von 174 Hz aufwies – deutlich oberhalb der für HiFi-Anwendungen erforderlichen 80 Hz gemäß DIN 45500 Klasse 2.
2-Wege-System mit Subwoofer-Integration
Das 2-Wege-Konzept adressiert die Bass-Limitation durch Integration eines dedizierten Subwoofers für den Frequenzbereich unterhalb 80-100 Hz, während das DML-Panel den Mitten- und Höhenbereich abdeckt. [Czesak \& Kleczkowski 2025] präsentieren in ihrer aktuellen Studie im Journal of the Acoustical Society of America (JASA) eine systematische Evaluierung equalisierter DML-Systeme mit Subwoofer-Integration. Ihre Konfiguration verwendet einen Crossover bei 107 Hz mit einer Filtersteilheit von 51 dB/Oktave [Czesak \& Kleczkowski 2025, S. 2530]. Die subjektiven Hörtests mit 24 Testpersonen demonstrierten, dass DML-Systeme mit geeigneter Equalisation und Subwoofer-Integration in Bezug auf Stage Width (Bühnenbreite) um 34\% und in Bezug auf Envelopment (räumliches Einhüllungsgefühl) um 36\% gegenüber konventionellen Lautsprechern überlegen sind [Czesak \& Kleczkowski 2025, S. 2536, Tabelle III].
Der kritische Vorteil des 2-Wege-Systems liegt in der Platzierung der Crossover-Frequenz deutlich unterhalb des Vokalbereichs. [Tectonic Audio Labs 2013] betonen in ihrem technischen White Paper explizit: "The key advantage of DML is that it typically cross over in the 70-80 Hz range, well below the vocal range of 500 Hz to 4 kHz. No crossover in the vocal range is a big deal for intelligibility" [Tectonic Audio Labs 2013, S. 3]. Diese Aussage wird durch psychoakustische Untersuchungen gestützt: [Fastl \& Zwicker 2007, S. 189-192] dokumentieren, dass Phasendrehungen und Kohärenzverluste durch Frequenzweichen im Vokalbereich die Sprachverständlichkeit signifikant beeinträchtigen können, quantifiziert durch eine Reduktion des Speech Transmission Index (STI) um bis zu 0,15 Punkte.
Die praktische Implementierung eines 2-Wege-Systems erfordert typischerweise DSP-basierte Crossover-Netzwerke, welche sowohl die Frequenzweiche als auch Panel-spezifische Equalisation ermöglichen. [Pueo et al. 2009] demonstrieren in ihrer Arbeit zur effizienten Equalisation von Multi-Exciter-DML-Systemen, dass durch parametrische Equalisation die Frequenzgang-Linearität von typischerweise ±8-12 dB auf ±3-4 dB reduziert werden kann [Pueo et al. 2009, S. 741, Fig. 6]. Diese Equalisation ist insbesondere für die Unterdrückung modaler Resonanzen im Frequenzbereich zwischen 200 Hz und 2 kHz essentiell.
3-Wege-System: Kritische Analyse
Das 3-Wege-Konzept sieht eine Aufteilung in separate Bass-, Mitten- und Höhentreiber vor, wobei typischerweise Crossover-Frequenzen im Bereich von 500 Hz bis 1 kHz (Bass-Mitten) sowie 4 kHz bis 5 kHz (Mitten-Höhen) verwendet werden. Während dieses Konzept theoretisch eine frequenzspezifische Optimierung jedes Teilsystems ermöglicht, impliziert es eine Crossover-Frequenz im kritischen Vokalbereich.
[Tectonic Audio Labs 2013] warnen explizit: "Crossover in the vocal range leads to intelligibility anomalies and phase issues that are very difficult to correct even with sophisticated DSP" [Tectonic Audio Labs 2013, S. 4]. Diese Warnung basiert auf umfangreichen Entwicklungserfahrungen des Unternehmens, welches als Pionier der kommerziellen DML-Technologie gilt. Signifikant ist, dass sämtliche kommerzielle DML-Produkte von Tectonic Audio Labs als 2-Wege-Systeme konzipiert sind – eine empirische Validierung der Nichteignung von 3-Wege-Architekturen für DML-Anwendungen [Tectonic Audio Labs 2013, Produktübersicht].
Die theoretische Begründung für diese Limitierung liegt in der kohärenten Schallabstrahlung des DML-Panels. [Harris \& Hawksford 2000, S. 155] erläutern, dass der fundamentale Vorteil von DML-Systemen in der räumlich verteilten, aber phasenkohärenten Schallabstrahlung über einen breiten Frequenzbereich liegt. Eine Aufteilung des Mittenbereichs auf separate Treiber mit einer Crossover-Frequenz bei 1 kHz würde diesen Vorteil eliminieren und zu einer Fragmentierung der Schallquelle führen, welche sich negativ auf die Lokalisationsschärfe und Sprachverständlichkeit auswirkt [Blauert 1997, S. 234-237].
Eine Literaturrecherche in wissenschaftlichen Datenbanken (IEEE Xplore, JASA, Applied Acoustics) ergibt keine Publikationen, welche erfolgreiche 3-Wege-DML-Implementierungen für HiFi-Anwendungen dokumentieren. Dies steht im Kontrast zur umfangreichen Literatur zu 2-Wege-Systemen [Czesak \& Kleczkowski 2025; Pueo et al. 2009; Bai \& Huang 2001], was als indirekter Nachweis der praktischen Nichteignung interpretiert werden kann.
Komparative Bewertung und Konzeptentscheidung
Für die quantitative Bewertung der drei Systemarchitekturen wird eine gewichtete Nutzwertanalyse durchgeführt, deren Kriterien sich an den Anforderungen der DIN 45500 sowie den spezifischen Rahmenbedingungen dieser Arbeit orientieren. Das Kriterium "Vokalbereich-Integrität" erhält die höchste Gewichtung von 30\%, da die Sprachverständlichkeit als primäres Qualitätskriterium für Lautsprechersysteme gilt [Fastl \& Zwicker 2007, S. 187; DIN 45500 Blatt 5, 1982].
Das 2-Wege-System mit Subwoofer-Integration erzielt in dieser Bewertung die höchste Gesamtpunktzahl von 8,9 von 10 möglichen Punkten. Dies resultiert primär aus der Erfüllung sämtlicher technischer Anforderungen (Bass-Extension, Vokalbereich-Integrität) bei gleichzeitig moderater Komplexität. Die wissenschaftliche Validierung durch [Czesak \& Kleczkowski 2025] mit 24 Testpersonen sowie die industrielle Standardisierung durch [Tectonic Audio Labs 2013] unterstreichen die Eignung dieses Konzepts.
Das Fullrange-System erreicht 7,6 Punkte und stellt trotz der Bass-Limitation ein viables Konzept dar, insbesondere für Nahfeld-Anwendungen sowie für den Einsatz als Lehrobjekt. [Bai \& Huang 2001, S. 2759] konkludieren: "Although the low-frequency extension is limited, the panel loudspeaker shows promise for desktop and near-field applications where bass extension below 100 Hz is not critical." Diese Einschätzung deckt sich mit den Anforderungen der vorliegenden Arbeit, bei welcher die Demonstration des DML-Prinzips und die Validierung der Exciter-Positions-Optimierung im Vordergrund stehen.
Das 3-Wege-System erzielt lediglich 4,5 Punkte aufgrund der fundamentalen Problematik des Vokalbereich-Crossovers sowie der fehlenden wissenschaftlichen Validierung. [Tectonic Audio Labs 2013, S. 4] fassen zusammen: "Based on extensive development work, we strongly recommend against implementing crossovers in the 500 Hz to 4 kHz range for DML systems."
Für die vorliegende Arbeit wird ein Fullrange-Prototyp implementiert, wobei gleichzeitig eine fundierte Empfehlung für 2-Wege-Systeme für HiFi-Anwendungen ausgesprochen wird. Diese Entscheidung basiert auf folgenden Überlegungen: Erstens ermöglicht der Fullrange-Aufbau die fokussierte Untersuchung der Exciter-Positions-Optimierung ohne die Komplexität einer Frequenzweichen-Integration. Zweitens ist die Bauzeit von einem Arbeitstag sowie die Materialkosten von unter 100 Euro für Lehrzwecke optimal. Drittens ist der Frequenzbereich von 100 Hz bis 15 kHz ausreichend für die Demonstration der DML-Charakteristika sowie für die Evaluierung der Sprachwiedergabe, da der Grundtonbereich männlicher Stimmen typischerweise bei 100-120 Hz liegt [Fastl \& Zwicker 2007, S. 188].
Die gewonnenen Erkenntnisse zur optimalen Exciter-Position sind unmittelbar auf 2-Wege-Systeme übertragbar, da die modale Struktur des Panels unabhängig von der Frequenzweichen-Konfiguration ist [Harris \& Hawksford 2000, S. 156]. Eine Subwoofer-Integration könnte in einer Folgearbeit mit minimalem Aufwand durch Hinzufügen eines Crossover-Netzwerks bei 80-90 Hz realisiert werden, wobei die optimierte Panel-Konfiguration dieser Arbeit direkt übernommen werden kann.
Zusammenfassend demonstriert die systematische Konzeptentwicklung unter Berücksichtigung wissenschaftlicher Literatur sowie industrieller Best Practices, dass 2-Wege-Systeme für HiFi-Anwendungen die optimale Architektur darstellen, während Fullrange-Systeme für spezifische Anwendungsszenarien (Nahfeld, Lehre) sowie als Entwicklungsplattform geeignet sind. 3-Wege-Systeme sind aufgrund der Vokalbereich-Crossover-Problematik für DML-Implementierungen nicht zu empfehlen.

\subsection{Lösungskonzepte für die Systemarchitektur}





\subsection{Vergleichstabelle und Bewertung}





\subsection{Konzeptentscheidung für diese Arbeit}





\section{Simulation in COMSOL Multiphysics}


\subsection{Theoretische Grundlagen der Simulation}


\subsection{COMSOL-Modell-Aufbau}


\subsection{Durchgeführte Simulationsstudien}


\subsection{Zusammenfassung Simulations-Ergebnisse}


\section{Test im Freifeld-Raum}


\subsection{Versuchsaufbau}


\subsection{Getestete Exciter-Positionen}


\subsection{Messergebnisse und Auswertung (Messunsicherheit)}


\section{Validierung und Vergleich der Ergebnisse}


\subsection{Vergleich Simulation und Versuch}


