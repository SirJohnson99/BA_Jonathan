\chapter{Einleitung}

Platzhalter Einleitung

\section{Problemstellung}

Konventionelle Membranlautsprecher weisen inherente Limitationen auf: gerichtete Schallabstrahlung mit begrenztem Sweet Spot sowie voluminöse Gehäuse mit Bautiefen über 20 cm limitieren Designfreiheit und Raumintegration.
Distributed-Mode-Loudspeaker (DML) bieten als Alternative omnidirektionale Abstrahlung (150-170°) und flache Bauformen (<2 cm) durch Biegewellen-Anregung in Platten. Dies eröffnet Anwendungen in Wandintegration, Automotive-Audio und architektonischen Installationen. Die zentrale Herausforderung besteht jedoch in Frequenzgang-Unregelmäßigkeiten von ±8-12 dB (Bai & Huang 2001), welche aus komplexer modaler Struktur resultieren. Die akustische Performance wird maßgeblich durch die Exciter-Position determiniert.
Trotz existierender Hersteller-Empfehlungen (z.B. Dayton Audio „2/5×3/5-Regel") fehlt eine systematische, FEM-gestützte Optimierung für spezifische Panel-Materialien wie Kappa-Sandwichplatten. Zudem besteht Forschungsbedarf bezüglich der optimalen Systemarchitektur (Fullrange, 2-Wege, 3-Wege) unter Berücksichtigung der Vokalbereich-Integrität (500-4000 Hz).
Für die Ostfalia Hochschule ergibt sich die Relevanz, durch Integration alternativer Schallwandler-Konzepte das Lehrangebot zu erweitern und die Innovationskompetenz angehender Ingenieure zu fördern. Es besteht somit der Bedarf an systematischer DML-Optimierung mittels Simulation und Experiment sowie der Entwicklung eines kostengünstigen, reproduzierbaren Prototyps für Lehrzwecke.


\section{Zielsetzung und Vorgehensweise}
Die vorliegende Arbeit verfolgt sowohl technische als auch pädagogische Zielsetzungen. Im technischen Kontext gilt es, einen funktionsfähigen Flachlautsprecher auf Basis der Distributed-Mode-Loudspeaker-Technologie (DML) zu entwickeln, welcher den Anforderungen der DIN 45500 für HiFi-Lautsprecher entspricht. Parallel dazu soll die Arbeit zur Erweiterung des Lehrangebots der Ostfalia Hochschule für angewandte Wissenschaften beitragen, indem Studierenden eine erweiterte Varietät von Akustik-Modulen geboten wird, um das technische Verständnis zu schärfen sowie Innovationskraft und Pioniergeist der zukünftigen Ingenieurgeneration zu fördern.

Im Zentrum der technischen Zielsetzung steht die systematische Optimierung der Exciter-Positionierung auf einem Flachpanel. Die Positionierung der Körperschallwandler beeinflusst maßgeblich die Anregung der Schwingungsmoden und determiniert folglich den resultierenden Frequenzgang. Diese Arbeit adressiert die bestehende Forschungslücke durch einen integrierten methodischen Ansatz, welcher experimentelle Freifeld-Messungen, Finite-Elemente-Simulationen in COMSOL Multiphysics sowie den Bau eines funktionsfähigen Prototyps kombiniert.

Ein weiteres zentrales Forschungsziel besteht in der systematischen Bewertung verschiedener Systemarchitekturen welche sich in Fullrange, 2-Wege und 3-Wege unterteilen lässt. Unter besonderer Berücksichtigung von Frequenzweichen im kritischen Vokalbereich 500 Hz bis 4 kHz. Der Vergleich mit konventionellen Membranlautsprechern erfolgt auf objektiver Ebene sowie durch subjektive Hörtests.

%Die methodische Vorgehensweise orientiert sich an der VDI-Richtlinie 2221 für systematische Produktentwicklung und wird durch den PDCA-Zyklus, dieser steht für Plan-Do-Check-Act, nach Deming iterativ optimiert. Die beiden genannten Methoden werden in den Grundlagen Kapitel 2.4 und 2.5 ausfürlicher beschieben. Die Kombination beider Ansätze gewährleistet eine nachvollziehbare, wissenschaftlich fundierte und reproduzierbare Vorgehensweise. Die PLAN-Phase umfasst die Anforderungsdefinition gemäß DIN 45500 sowie die Versuchsplanung. In der DO-Phase erfolgen Simulation in COMSOL sowie die experimentelle Freifeld-Messungen und Prototypenbau. Die CHECK-Phase beinhaltet die Validierung der Simulation gegenüber experimentellen Ergebnissen. In der ACT-Phase werden basierend auf den Erkenntnissen der finale Prototyp gebaut und subjektive Hörtests durchgeführt.

%Die technischen Anforderungen orientieren sich an DIN 45500 Klasse 2: Übertragungsbereich 80 Hz bis 12,5 kHz (±10 dB), Frequenzgang-Linearität ±6 dB ohne Equalizer (200 Hz - 10 kHz), Schalldruck >80 dB SPL bei 1 m Abstand, Klirrfaktor <5% bei 90 dB SPL, Panel-Größe <1 m² sowie Materialkosten <100 Euro zur Gewährleistung der Nachbaubarkeit für Studierende.

Übergeordnet leistet diese Arbeit einen Beitrag zur Erweiterung des an der Ostfalia verfügbaren Spektrums von Lautsprechertechnologien und bietet angehenden Ingenieuren einen fundierten Einblick in alternative Schallwandler-Konzepte. Durch die Auseinandersetzung mit innovativen Technologien wird die Innovationskompetenz und das kritische analytische Denken der Studierenden gefördert, was für ihre zukünftige Tätigkeit als Ingenieure von fundamentaler Bedeutung ist.