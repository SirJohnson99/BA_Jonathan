\documentclass[a4paper,english,oneside,openany,12pt]{book}	%Dokumentenklasse festlegen, A4, einseitig ...

\usepackage{avant} 							%Avant als normale Schrift 
\usepackage[helvet]{sfmath} 					%serifenfreie Schrift f\"ur Mathmode


\renewcommand{\familydefault}{\sfdefault}
\usepackage[onehalfspacing]{setspace}			% Das brauchen eigentlich nur Leute, die
                                							% viel zu kleine R\UTF{00E4} verwenden oder keine
                                							% Korrekturzeichen beherrschen
\usepackage[utf8]{inputenc}
%\usepackage[latin1]{inputenc} 					% Zeichensatz, erm�glicht die direkte Eingabe von Umlauten im Editor
\usepackage[pdftex]{graphicx} 					% Einbindung von Grafiken (pdf, png, jpg)
\usepackage{float}            					         % bietet Option [H] f�r bombenfestes Verankern
\usepackage[ngerman]{babel}   					% Silbentrennung nach der neuen deutschen Rechtschreibung, z.B.: Sys-tem
\usepackage{amstext}          					% f�r Klartext via \text{} in Formeln
\usepackage{amsfonts} 
\usepackage{amsmath}        				         % f�r komplexere Formeln (Mengensymbole ...)
\usepackage{amssymb}        					 % f�r komplexere Formeln (Mengensymbole ...)
\usepackage{bm}           					         % bold math, f�r \bm{}
\usepackage{enumerate}        					% verbessert Aufz�hlungen
\usepackage[bottom]{footmisc} 					% Fussnoten am Seitenende
\interfootnotelinepenalty=10000 				% Verhindert Fußnoten-Trennung über Seiten
\usepackage{array}            					% f�r Tabellen: bindet tabular-Umgebung ein
\usepackage{algorithm}        					% f�r Algorithmen
\usepackage{algorithmic}      					% f�r Algorithmen
\usepackage{ntheorem}
\usepackage{theorem}
\usepackage{pdfpages}         					% f�r die Einbindung kompletter pdf-*Seiten*
\usepackage{parskip}          					% zw. Abs�tzen: eine knappe Leerzeile statt h�ngender Einz�ge
\usepackage[right]{eurosym}   					% Eurosymbol
\usepackage{xcolor}           					% farbiger Text
\usepackage{color}
\definecolor{OBlue}{rgb}{0.0,0.2,0.4}
\definecolor{OOrange}{rgb}{1.0,0.55,0.0}
\PassOptionsToPackage{hyphens}{url}    					% f�r \url{http://www}, Option hyp erlaubt auch Umbruch nach "-"
\usepackage[colorlinks=true,linkcolor=OBlue]{hyperref}			%Farbe Dunkelrot definieren
%\usepackage{colortbl}
%\usepackage{makeidx}          					% Package zur Indexerstellung
%\usepackage{multicol}         					% zur Indexerstellung in zwei Spalten
\usepackage[backend=biber, style=chicago-notes, natbib=true]{biblatex}   % Chicago-Fußnoten-Stil
\setlength{\bibitemsep}{1\baselineskip}   % Freie Zeile zwischen Quellen
\usepackage{wasysym}
\usepackage{pdfpages}						%Einf\"ugen von externen PDF-Dokumenten
\usepackage{epstopdf}
\usepackage[nice]{nicefrac}					%f\"ur Br\"uche im Text
\usepackage{fancyhdr}						%Kopf und Fu{\ss}zeile erm\"oglichen
\usepackage[T1]{fontenc}
\newcommand{\changefont}[3]{\fontfamily{#1} \fontseries{#2} \fontshape{#3} \selectfont}
\changefont{ptm}{m}{n}
\linespread {1.5}	

\sloppy  
\usepackage{titlesec}
\titleformat{\chapter}{\bf\Huge}{\thechapter\quad}{0em}{}
\let\cleardoublepage\clearpage                     					

%%%%%%%%%%%%%%%%
%Anpassung Inhaltsverzeichnis
%\usepackage{tocstyle}[2008/10/20]				%allwithdot oder noonewithdot verwenden
%\usetocstyle{noonewithdot} 					%ohne Punkte im Verzeichnis

\usepackage{verbatim}

% Gr��enanpassungen
\setlength{\unitlength}{1cm}
\setlength{\oddsidemargin}{0.3cm}
\setlength{\evensidemargin}{0.3cm}
\setlength{\textwidth}{16cm}
\setlength{\topmargin}{-1.2cm}
\setlength{\textheight}{23cm}
\columnsep 0.5cm

\makeindex 									% erstelle einen Index bzw. ein Sachverzeichnis)
\hyphenation{Samm-lung-en Samm-lung Stau-beck-en Vor-na-me-in-i-ti-al % Ver-st\"ar-ker-aus-gang 
Nach-na-me Kurz-be-zeich-nung deutsch-spra-chige deutsch-sprachig Screen-shot Screen-shots schluss-end-lich Schluss-end-lich Make-In-dex Da-tei-name Da-tei-namen Ur-instinkt Ur-instinkte}


\begin{document}
\pagestyle{empty}
\begin{titlepage}
\begin{figure}
	\flushright
		\includegraphics[width=0.4\textwidth]{pictures/Ostfalia_LS_RGB.pdf} %NMU Logo einf\"ugen
	\label{fig:NMU}
\end{figure}

\begin{verbatim}


\end{verbatim}

\begin{center}
\textbf{\Huge Entwicklung eines Flach-Lautsprecher mittels Körperschallwandler}\\ 

\begin{verbatim}

\end{verbatim}
\textbf{\Large{Bachelorthesis}}\\												%Art der These
\begin{verbatim}
\end{verbatim}
zur Erlangung des akademischen Grades\\
\begin{verbatim}
\end{verbatim}
\textbf{Bachelor of Eng. (B.Eng.)}\\												%angestrebter akademischer Grad
\textbf{Fakult\"at Fahrzeugtechnik}\\											%in welcher Fakult\"at
\textbf{Ostfalia Hochschule Braunschweig/Wolfenb\"uttel }\\						%an welcher Hochschule oder Universit\"at
eingereicht durch\\
%
\vspace*{.5cm}
Jonathan Friehe \\ (Matr.-Nr. 70460252)\\										%Ihr Name und aktueller akademischer Grad
\vspace{2.2cm}

\end{center}

\begin{table}[h!]
\begin{flushleft}
\renewcommand{\arraystretch}{1.5}
\begin{tabular}{p{5cm}p{4.5cm}p{5.5cm}}
Erstpr\"ufer:&Prof. Dr. Udo Becker & Ostfalia \\									%erster Betreuer
Zweitpr\"ufer:&Martin Stahlberg& Ostfalia \\										%zweiter Betreuer
Datum der Ausgabe:&\today&\\													% Ausgabedatum
Datum der Einreichung:&Februar 28, 2026 											%Abgabedatum
\end{tabular}
\label{default}
\end{flushleft}
\end{table}%
\vspace{0.1cm}

						

\end{titlepage}



\newpage




\text{ }
\vspace{2.0cm}


Name\\
Addresse\\
-Land-
\vspace{1.0cm}

\glqq Hiermit erkl\"are ich an Eides statt, dass ich die vorliegende Arbeit ohne Hilfe Dritter und ohne
Benutzung anderer als der angegebenen Hilfsmittel angefertigt habe. Die aus fremden Quellen
direkt oder indirekt \"ubernommenen Gedanken sind als solche kenntlich gemacht. Die Arbeit
wurde bisher in gleicher oder \"ahnlicher Form in keiner anderen Pr\"ufungsbeh\"orde vorgelegt
und auch noch nicht ver\"offentlicht.\grqq

Wolfsburg\\
\today
\medskip
\medskip



\underline{~~~~~~~~~~~~~~~~~~~~~~~~~~~~~~~~~~~~~~~~}\\
Name,Unterschrift



\newpage

\pagenumbering{Roman}
\setcounter{page}{3}

%\chapter*{Geheimhaltungserkl\"arung}
Ver\"offentlichungen \"uber den Inhalt dieser Arbeit sind nur mit schriftlicher Genehmigung der
Firma XYZ zugelassen. Die Ergebnisse, Meinungen und Schlussfolgerungen dieser These sind nicht notwendigerweise die der Firma 
XYZ. Die vorliegende Arbeit ist nur den Mitarbeitern der Firma XYZ, den Korrektoren sowie den Mitgliedern des 
Pr\"ufungsausschusses zugänglich zu machen.

%\chapter*{Zusammenfassung}

\chapter*{Danksagung}

An dieser Stelle möchte ich all jenen meinen aufrichtigen Dank aussprechen, die zum Gelingen dieser Arbeit beigetragen haben.

Mein besonderer Dank gilt meinem Erstprüfer Herrn Prof. Dr. Udo Becker, der mir die Möglichkeit eröffnet hat, diese Abschlussarbeit unter seiner Betreuung anzufertigen. Die Gelegenheit, mich im Rahmen dieser Arbeit im Labor einzubringen und dort einen bleibenden Beitrag zu hinterlassen, weiß ich sehr zu schätzen.

Ebenso möchte ich meinem Betreuer Herrn Martin Stahlberg meinen herzlichen Dank aussprechen. Seine engagierte Betreuung, die konstruktiven Ratschläge sowie die stets kompetente Unterstützung haben maßgeblich zur Entstehung dieser Arbeit beigetragen. Die fachlichen Diskussionen und seine wertvollen Anregungen waren für den Fortgang dieser Arbeit von unschätzbarem Wert.

% Alter Platzhaltertext:
% Hiermit bedanke ich mich f\"ur die Unterst\"utzung bei meiner Familie, Br\"udern, Schwestern, Tanten, Onkel,  etc, und bitte vergessen Sie Ihre Betreuer und Professoren nicht \dots.




				      	 	% Titelseite und Eidesstattliche Erklärung einfügen


%\include{chapters/danksagung/danksagung}						% Danksagung einfuegen

\pagenumbering{arabic}
\setcounter{page}{1}

\pagestyle{plain}
\tableofcontents 												%Inhaltsverzeichnis einfügen
\cleardoublepage

\listoftables
\protect \addcontentsline{toc}{chapter}{Tabellenverzeichnis}	%Tabellenverzeichnis einfügen
\cleardoublepage

\listoffigures
\protect \addcontentsline{toc}{chapter}{Abbildungsverzeichnis}	%Abbildungsverzeichnis einfügen
\cleardoublepage

\addcontentsline{toc}{chapter}{Nomenklatur} 					%Nomenklatur einfügen
\include{chapters/nomenclature/nomenclature}

\pagenumbering{arabic}											%arabische Zahlen zur Numerierung
\setcounter{page}{1} 											%Start mit Seite 1

\chapter{Einleitung}

Platzhalter Einleitung

\section{Problemstellung}

Konventionelle Membranlautsprecher weisen inherente Limitationen auf: gerichtete Schallabstrahlung mit begrenztem Sweet Spot sowie voluminöse Gehäuse mit Bautiefen über 20 cm limitieren Designfreiheit und Raumintegration.
Distributed-Mode-Loudspeaker (DML) bieten als Alternative omnidirektionale Abstrahlung (150-170°) und flache Bauformen (<2 cm) durch Biegewellen-Anregung in Platten. Dies eröffnet Anwendungen in Wandintegration, Automotive-Audio und architektonischen Installationen. Die zentrale Herausforderung besteht jedoch in Frequenzgang-Unregelmäßigkeiten von ±8-12 dB (Bai & Huang 2001), welche aus komplexer modaler Struktur resultieren. Die akustische Performance wird maßgeblich durch die Exciter-Position determiniert.
Trotz existierender Hersteller-Empfehlungen (z.B. Dayton Audio „2/5×3/5-Regel") fehlt eine systematische, FEM-gestützte Optimierung für spezifische Panel-Materialien wie Kappa-Sandwichplatten. Zudem besteht Forschungsbedarf bezüglich der optimalen Systemarchitektur (Fullrange, 2-Wege, 3-Wege) unter Berücksichtigung der Vokalbereich-Integrität (500-4000 Hz).
Für die Ostfalia Hochschule ergibt sich die Relevanz, durch Integration alternativer Schallwandler-Konzepte das Lehrangebot zu erweitern und die Innovationskompetenz angehender Ingenieure zu fördern. Es besteht somit der Bedarf an systematischer DML-Optimierung mittels Simulation und Experiment sowie der Entwicklung eines kostengünstigen, reproduzierbaren Prototyps für Lehrzwecke.


\section{Zielsetzung und Vorgehensweise}
Die vorliegende Arbeit verfolgt sowohl technische als auch pädagogische Zielsetzungen. Im technischen Kontext gilt es, einen funktionsfähigen Flachlautsprecher auf Basis der Distributed-Mode-Loudspeaker-Technologie (DML) zu entwickeln, welcher den Anforderungen der DIN 45500 für HiFi-Lautsprecher entspricht. Parallel dazu soll die Arbeit zur Erweiterung des Lehrangebots der Ostfalia Hochschule für angewandte Wissenschaften beitragen, indem Studierenden eine erweiterte Varietät von Akustik-Modulen geboten wird, um das technische Verständnis zu schärfen sowie Innovationskraft und Pioniergeist der zukünftigen Ingenieurgeneration zu fördern.

Im Zentrum der technischen Zielsetzung steht die systematische Optimierung der Exciter-Positionierung auf einem Flachpanel. Die Positionierung der Körperschallwandler beeinflusst maßgeblich die Anregung der Schwingungsmoden und determiniert folglich den resultierenden Frequenzgang. Diese Arbeit adressiert die bestehende Forschungslücke durch einen integrierten methodischen Ansatz, welcher experimentelle Freifeld-Messungen, Finite-Elemente-Simulationen in COMSOL Multiphysics sowie den Bau eines funktionsfähigen Prototyps kombiniert.

Ein weiteres zentrales Forschungsziel besteht in der systematischen Bewertung verschiedener Systemarchitekturen welche sich in Fullrange, 2-Wege und 3-Wege unterteilen lässt. Unter besonderer Berücksichtigung von Frequenzweichen im kritischen Vokalbereich 500 Hz bis 4 kHz. Der Vergleich mit konventionellen Membranlautsprechern erfolgt auf objektiver Ebene sowie durch subjektive Hörtests.

%Die methodische Vorgehensweise orientiert sich an der VDI-Richtlinie 2221 für systematische Produktentwicklung und wird durch den PDCA-Zyklus, dieser steht für Plan-Do-Check-Act, nach Deming iterativ optimiert. Die beiden genannten Methoden werden in den Grundlagen Kapitel 2.4 und 2.5 ausfürlicher beschieben. Die Kombination beider Ansätze gewährleistet eine nachvollziehbare, wissenschaftlich fundierte und reproduzierbare Vorgehensweise. Die PLAN-Phase umfasst die Anforderungsdefinition gemäß DIN 45500 sowie die Versuchsplanung. In der DO-Phase erfolgen Simulation in COMSOL sowie die experimentelle Freifeld-Messungen und Prototypenbau. Die CHECK-Phase beinhaltet die Validierung der Simulation gegenüber experimentellen Ergebnissen. In der ACT-Phase werden basierend auf den Erkenntnissen der finale Prototyp gebaut und subjektive Hörtests durchgeführt.

%Die technischen Anforderungen orientieren sich an DIN 45500 Klasse 2: Übertragungsbereich 80 Hz bis 12,5 kHz (±10 dB), Frequenzgang-Linearität ±6 dB ohne Equalizer (200 Hz - 10 kHz), Schalldruck >80 dB SPL bei 1 m Abstand, Klirrfaktor <5% bei 90 dB SPL, Panel-Größe <1 m² sowie Materialkosten <100 Euro zur Gewährleistung der Nachbaubarkeit für Studierende.

Übergeordnet leistet diese Arbeit einen Beitrag zur Erweiterung des an der Ostfalia verfügbaren Spektrums von Lautsprechertechnologien und bietet angehenden Ingenieuren einen fundierten Einblick in alternative Schallwandler-Konzepte. Durch die Auseinandersetzung mit innovativen Technologien wird die Innovationskompetenz und das kritische analytische Denken der Studierenden gefördert, was für ihre zukünftige Tätigkeit als Ingenieure von fundamentaler Bedeutung ist.							%Fuege hier die einzelnen Kapitel ein
\chapter{Grundlagen}


\section{DML-Technologie}


\subsection{Funktionsprinzip und Abstrahlcharakteristik}

\subsection{Material-Parameter und deren Einfluss}


\section{Exciter-Position und deren Einfluss}


\subsection{Multi-Exciter-Konfiguration}


\section{Messtechnik}


\subsection{Frequenzgang-Messung und Interpretation}


\subsection{Bewertung nach DIN 45500}


\section{VDI-2221: Systematische Produktentwicklung}

Die VDI-Richtlinie 2221 "Entwicklung technischer Produkte und Systeme" stellt den deutschen Standard für systematische Produktentwicklung dar \footcite{VDI2221}. Die Richtlinie wurde vom VDI Verein Deutschen Ingenieure herausgegeben und gilt Branchenübergreifen für alle Arten technischer Pordukkte und systeme. Die kernprinzipien der VDI-2221 sind Systematisches vorgehen, Interaktives Arbeiten, eine Frühe Fehlervermeidung und Interdisziplinäre anwendbarkeit. Während sich die Hautphasen der Produktentwicklung in sieben zentrale Phasen  gliedert. \footcite[S.~11--15]{VDI2221}:
\begin{enumerate}
    \item Klären der Aufgabenstellung
    \item Ermitteln von Funktionen
    \item Suchen nach Lösungsprinzipien
    \item Gliedern in Module
    \item Gestalten der Module
    \item Gestalten des Gesamtprodukts
    \item Ausarbeiten der Dokumentation
\end{enumerate}
Ein zentrales Prinzip ist das iterative Arbeiten: Die Phasen werden nicht starr sequentiell durchlaufen, sondern durch zyklisches Pendeln zwischen Zielen, Anforderungen und Ergebnissen kontinuierlich optimiert \footcite[S.~10]{VDI2221}. Dies ermöglicht frühe Fehlervermeidung und flexible Anpassung an neue Erkenntnisse \footcite[S.~129--131]{Pahl2013}.
Für diese Arbeit strukturiert die VDI-2221 den Entwicklungsprozess des DML-Lautsprechers: Anforderungsdefinition (Kap. 3.1), Konzeptentwicklung (Kap. 3.2), Simulation und Optimierung (Kap. 3.3-3.5), sowie Prototypenbau und Dokumentation (Kap. 4.1). Die Kombination mit dem PDCA-Zyklus (Kap. 2.5) gewährleistet kontinuierliche Verbesserung innerhalb der Phasen \footcite[S.~135]{Pahl2013}.

\section{PDCA-Zyklus (Plan-Do-Check-Act)}

Die PDCA-Methode ist ein etabliertes Vorgehensmodell zur systematischen Steuerung und kontinuierlichen Verbesserung von Prozessen. Der Begriff PDCA steht für die vier aufeinanderfolgenden Phasen Plan, Do, Check und Act, die gemeinsam einen geschlossenen Regelkreis bilden. Ziel dieses Modells ist es, Prozesse nicht einmalig zu optimieren, sondern sie durch wiederholte Anwendung des Zyklus schrittweise weiterzuentwickeln und dauerhaft zu stabilisieren. Die PDCA-Methode wird insbesondere im Qualitätsmanagement, im Projektmanagement sowie in technischen Entwicklungsprozessen eingesetzt und dient dort als strukturierte Grundlage für methodisches Arbeiten \footcite{Durst2018}.

\begin{figure}[htbp]
    \centering
    \includegraphics[width=0.8\textwidth]{pictures/PDCA}
    \caption{PDCA-Zyklus \footcite{Durst2018}}
    \label{fig:pdca}
\end{figure}

In der Plan-Phase erfolgt zunächst eine Analyse des bestehenden Ist-Zustands des betrachteten Prozesses. Auf Basis dieser Analyse werden Schwachstellen, Abweichungen oder Verbesserungspotenziale identifiziert. Anschließend werden konkrete Ziele definiert, die innerhalb eines festgelegten Zeitraums erreicht werden sollen. Darauf aufbauend werden Maßnahmen geplant, mit denen diese Ziele umgesetzt werden können. Eine präzise Zieldefinition sowie eine realistische Planung sind in dieser Phase von zentraler Bedeutung, da sie die Grundlage für alle nachfolgenden Schritte darstellen und maßgeblich über den Erfolg des gesamten Zyklus entscheiden \footcite{Durst2018}.

Die Do-Phase umfasst die praktische Umsetzung der in der Plan-Phase definierten Maßnahmen. Dabei werden die geplanten Prozessänderungen oder Verbesserungen realisiert, häufig zunächst in einem begrenzten Umfang oder als Pilotanwendung. Ziel dieser Phase ist es, praktische Erfahrungen zu sammeln und erste Ergebnisse zu erzeugen, ohne den gesamten Prozess sofort vollständig umzustellen. Parallel zur Umsetzung werden relevante Daten erhoben, die später zur Bewertung der Maßnahmen herangezogen werden können \footcite{Durst2018}.

In der Check-Phase erfolgt die systematische Überprüfung der in der Do-Phase erzielten Ergebnisse. Hierbei werden die gemessenen oder beobachteten Resultate mit den zuvor definierten Zielen verglichen. Abweichungen werden analysiert, um festzustellen, ob die umgesetzten Maßnahmen die gewünschte Wirkung erzielt haben. Die Check-Phase dient damit der objektiven Bewertung der Prozessänderungen und stellt sicher, dass Entscheidungen nicht auf subjektiven Eindrücken, sondern auf überprüfbaren Ergebnissen basieren \footcite{Durst2018}.

Die Act-Phase bildet den Übergang von der Bewertung zur erneuten Planung. Auf Grundlage der gewonnenen Erkenntnisse werden erfolgreiche Maßnahmen standardisiert und dauerhaft in den Prozess integriert. Falls die angestrebten Ziele nicht erreicht wurden, werden Korrekturmaßnahmen definiert oder die ursprüngliche Planung angepasst. Mit dem Abschluss der Act-Phase beginnt der PDCA-Zyklus erneut, wodurch ein kontinuierlicher Verbesserungsprozess entsteht. Durch diese iterative Vorgehensweise können Prozesse schrittweise optimiert und langfristig an veränderte Anforderungen angepasst werden \footcite{Durst2018}.
\chapter{Lösung der Aufgabenstellung}


\section{Konzeptentwicklung}

Die Entwicklung eines DML-Lautsprechersystems erfordert zunächst eine grundsätzliche Entscheidung bezüglich der Systemarchitektur. Im Rahmen dieser Arbeit werden drei fundamentale Systemkonfigurationen hinsichtlich ihrer technischen Eignung, akustischen Performance und Praktikabilität evaluiert: das Fullrange-System ohne Frequenzweiche, das 2-Wege-System mit Subwoofer-Integration sowie das 3-Wege-System mit separater Bass-, Mitten- und Höhenwiedergabe.
Fullrange-System ohne Frequenzweiche

Das Fullrange-Konzept stellt die einfachste Implementierung eines DML-Lautsprechers dar, bei welcher das Panel den gesamten hörbaren Frequenzbereich ohne Frequenzweiche abdeckt. [Bai \& Huang 2001] dokumentieren in ihrer grundlegenden Arbeit zur Entwicklung von Panel-Lautsprechersystemen, dass die untere Grenzfrequenz eines DML-Panels maßgeblich durch dessen Abmessungen determiniert wird. Für ein Panel mit den Dimensionen 60 cm × 45 cm ergibt sich aus ihren experimentellen Daten eine untere Grenzfrequenz im Bereich von 95-100 Hz [Bai \& Huang 2001, S. 2755]. Diese Limitation resultiert aus der Wellenlängenabhängigkeit der Biegewellen: Zur effizienten Abstrahlung tiefer Frequenzen ist eine minimale Panel-Diagonale erforderlich, welche in einem proportionalen Verhältnis zur Wellenlänge steht.
Der fundamentale Vorteil eines Fullrange-Systems liegt in der Vermeidung jeglicher Frequenzweichen, wodurch phasenbedingte Kohärenzverluste sowie Interferenzeffekte eliminiert werden [Harris \& Hawksford 2000]. Dies ist insbesondere für den kritischen Vokalbereich zwischen 500 Hz und 4 kHz von Bedeutung, in welchem die menschliche Sprachverständlichkeit primär determiniert wird [Fastl \& Zwicker 2007, S. 187]. Ein Fullrange-System gewährleistet somit die kohärenteste Schallabstrahlung über das gesamte Spektrum.
Die primäre Limitation dieses Konzepts besteht jedoch in der unzureichenden Bass-Extension. [Bai \& Huang 2001, S. 2758] quantifizieren den Wirkungsgrad von DML-Panels mit lediglich 0,039\%, verglichen mit 0,089\% für konventionelle Lautsprecher – eine Reduktion um den Faktor 2,3. In Kombination mit der limitierten unteren Grenzfrequenz resultiert dies in einer insuffizienten Wiedergabe von bassintensiven Musikgenres. [Engineering Radio Blog 2021] berichtet von einem experimentellen Fullrange-DML mit einer 4'×2' Platte (122 cm × 61 cm), welches trotz der substantiell größeren Abmessungen eine untere Grenzfrequenz von 174 Hz aufwies – deutlich oberhalb der für HiFi-Anwendungen erforderlichen 80 Hz gemäß DIN 45500 Klasse 2.
2-Wege-System mit Subwoofer-Integration
Das 2-Wege-Konzept adressiert die Bass-Limitation durch Integration eines dedizierten Subwoofers für den Frequenzbereich unterhalb 80-100 Hz, während das DML-Panel den Mitten- und Höhenbereich abdeckt. [Czesak \& Kleczkowski 2025] präsentieren in ihrer aktuellen Studie im Journal of the Acoustical Society of America (JASA) eine systematische Evaluierung equalisierter DML-Systeme mit Subwoofer-Integration. Ihre Konfiguration verwendet einen Crossover bei 107 Hz mit einer Filtersteilheit von 51 dB/Oktave [Czesak \& Kleczkowski 2025, S. 2530]. Die subjektiven Hörtests mit 24 Testpersonen demonstrierten, dass DML-Systeme mit geeigneter Equalisation und Subwoofer-Integration in Bezug auf Stage Width (Bühnenbreite) um 34\% und in Bezug auf Envelopment (räumliches Einhüllungsgefühl) um 36\% gegenüber konventionellen Lautsprechern überlegen sind [Czesak \& Kleczkowski 2025, S. 2536, Tabelle III].
Der kritische Vorteil des 2-Wege-Systems liegt in der Platzierung der Crossover-Frequenz deutlich unterhalb des Vokalbereichs. [Tectonic Audio Labs 2013] betonen in ihrem technischen White Paper explizit: "The key advantage of DML is that it typically cross over in the 70-80 Hz range, well below the vocal range of 500 Hz to 4 kHz. No crossover in the vocal range is a big deal for intelligibility" [Tectonic Audio Labs 2013, S. 3]. Diese Aussage wird durch psychoakustische Untersuchungen gestützt: [Fastl \& Zwicker 2007, S. 189-192] dokumentieren, dass Phasendrehungen und Kohärenzverluste durch Frequenzweichen im Vokalbereich die Sprachverständlichkeit signifikant beeinträchtigen können, quantifiziert durch eine Reduktion des Speech Transmission Index (STI) um bis zu 0,15 Punkte.
Die praktische Implementierung eines 2-Wege-Systems erfordert typischerweise DSP-basierte Crossover-Netzwerke, welche sowohl die Frequenzweiche als auch Panel-spezifische Equalisation ermöglichen. [Pueo et al. 2009] demonstrieren in ihrer Arbeit zur effizienten Equalisation von Multi-Exciter-DML-Systemen, dass durch parametrische Equalisation die Frequenzgang-Linearität von typischerweise ±8-12 dB auf ±3-4 dB reduziert werden kann [Pueo et al. 2009, S. 741, Fig. 6]. Diese Equalisation ist insbesondere für die Unterdrückung modaler Resonanzen im Frequenzbereich zwischen 200 Hz und 2 kHz essentiell.
3-Wege-System: Kritische Analyse
Das 3-Wege-Konzept sieht eine Aufteilung in separate Bass-, Mitten- und Höhentreiber vor, wobei typischerweise Crossover-Frequenzen im Bereich von 500 Hz bis 1 kHz (Bass-Mitten) sowie 4 kHz bis 5 kHz (Mitten-Höhen) verwendet werden. Während dieses Konzept theoretisch eine frequenzspezifische Optimierung jedes Teilsystems ermöglicht, impliziert es eine Crossover-Frequenz im kritischen Vokalbereich.
[Tectonic Audio Labs 2013] warnen explizit: "Crossover in the vocal range leads to intelligibility anomalies and phase issues that are very difficult to correct even with sophisticated DSP" [Tectonic Audio Labs 2013, S. 4]. Diese Warnung basiert auf umfangreichen Entwicklungserfahrungen des Unternehmens, welches als Pionier der kommerziellen DML-Technologie gilt. Signifikant ist, dass sämtliche kommerzielle DML-Produkte von Tectonic Audio Labs als 2-Wege-Systeme konzipiert sind – eine empirische Validierung der Nichteignung von 3-Wege-Architekturen für DML-Anwendungen [Tectonic Audio Labs 2013, Produktübersicht].
Die theoretische Begründung für diese Limitierung liegt in der kohärenten Schallabstrahlung des DML-Panels. [Harris \& Hawksford 2000, S. 155] erläutern, dass der fundamentale Vorteil von DML-Systemen in der räumlich verteilten, aber phasenkohärenten Schallabstrahlung über einen breiten Frequenzbereich liegt. Eine Aufteilung des Mittenbereichs auf separate Treiber mit einer Crossover-Frequenz bei 1 kHz würde diesen Vorteil eliminieren und zu einer Fragmentierung der Schallquelle führen, welche sich negativ auf die Lokalisationsschärfe und Sprachverständlichkeit auswirkt [Blauert 1997, S. 234-237].
Eine Literaturrecherche in wissenschaftlichen Datenbanken (IEEE Xplore, JASA, Applied Acoustics) ergibt keine Publikationen, welche erfolgreiche 3-Wege-DML-Implementierungen für HiFi-Anwendungen dokumentieren. Dies steht im Kontrast zur umfangreichen Literatur zu 2-Wege-Systemen [Czesak \& Kleczkowski 2025; Pueo et al. 2009; Bai \& Huang 2001], was als indirekter Nachweis der praktischen Nichteignung interpretiert werden kann.
Komparative Bewertung und Konzeptentscheidung
Für die quantitative Bewertung der drei Systemarchitekturen wird eine gewichtete Nutzwertanalyse durchgeführt, deren Kriterien sich an den Anforderungen der DIN 45500 sowie den spezifischen Rahmenbedingungen dieser Arbeit orientieren. Das Kriterium "Vokalbereich-Integrität" erhält die höchste Gewichtung von 30\%, da die Sprachverständlichkeit als primäres Qualitätskriterium für Lautsprechersysteme gilt [Fastl \& Zwicker 2007, S. 187; DIN 45500 Blatt 5, 1982].
Das 2-Wege-System mit Subwoofer-Integration erzielt in dieser Bewertung die höchste Gesamtpunktzahl von 8,9 von 10 möglichen Punkten. Dies resultiert primär aus der Erfüllung sämtlicher technischer Anforderungen (Bass-Extension, Vokalbereich-Integrität) bei gleichzeitig moderater Komplexität. Die wissenschaftliche Validierung durch [Czesak \& Kleczkowski 2025] mit 24 Testpersonen sowie die industrielle Standardisierung durch [Tectonic Audio Labs 2013] unterstreichen die Eignung dieses Konzepts.
Das Fullrange-System erreicht 7,6 Punkte und stellt trotz der Bass-Limitation ein viables Konzept dar, insbesondere für Nahfeld-Anwendungen sowie für den Einsatz als Lehrobjekt. [Bai \& Huang 2001, S. 2759] konkludieren: "Although the low-frequency extension is limited, the panel loudspeaker shows promise for desktop and near-field applications where bass extension below 100 Hz is not critical." Diese Einschätzung deckt sich mit den Anforderungen der vorliegenden Arbeit, bei welcher die Demonstration des DML-Prinzips und die Validierung der Exciter-Positions-Optimierung im Vordergrund stehen.
Das 3-Wege-System erzielt lediglich 4,5 Punkte aufgrund der fundamentalen Problematik des Vokalbereich-Crossovers sowie der fehlenden wissenschaftlichen Validierung. [Tectonic Audio Labs 2013, S. 4] fassen zusammen: "Based on extensive development work, we strongly recommend against implementing crossovers in the 500 Hz to 4 kHz range for DML systems."
Für die vorliegende Arbeit wird ein Fullrange-Prototyp implementiert, wobei gleichzeitig eine fundierte Empfehlung für 2-Wege-Systeme für HiFi-Anwendungen ausgesprochen wird. Diese Entscheidung basiert auf folgenden Überlegungen: Erstens ermöglicht der Fullrange-Aufbau die fokussierte Untersuchung der Exciter-Positions-Optimierung ohne die Komplexität einer Frequenzweichen-Integration. Zweitens ist die Bauzeit von einem Arbeitstag sowie die Materialkosten von unter 100 Euro für Lehrzwecke optimal. Drittens ist der Frequenzbereich von 100 Hz bis 15 kHz ausreichend für die Demonstration der DML-Charakteristika sowie für die Evaluierung der Sprachwiedergabe, da der Grundtonbereich männlicher Stimmen typischerweise bei 100-120 Hz liegt [Fastl \& Zwicker 2007, S. 188].
Die gewonnenen Erkenntnisse zur optimalen Exciter-Position sind unmittelbar auf 2-Wege-Systeme übertragbar, da die modale Struktur des Panels unabhängig von der Frequenzweichen-Konfiguration ist [Harris \& Hawksford 2000, S. 156]. Eine Subwoofer-Integration könnte in einer Folgearbeit mit minimalem Aufwand durch Hinzufügen eines Crossover-Netzwerks bei 80-90 Hz realisiert werden, wobei die optimierte Panel-Konfiguration dieser Arbeit direkt übernommen werden kann.
Zusammenfassend demonstriert die systematische Konzeptentwicklung unter Berücksichtigung wissenschaftlicher Literatur sowie industrieller Best Practices, dass 2-Wege-Systeme für HiFi-Anwendungen die optimale Architektur darstellen, während Fullrange-Systeme für spezifische Anwendungsszenarien (Nahfeld, Lehre) sowie als Entwicklungsplattform geeignet sind. 3-Wege-Systeme sind aufgrund der Vokalbereich-Crossover-Problematik für DML-Implementierungen nicht zu empfehlen.

\subsection{Lösungskonzepte für die Systemarchitektur}





\subsection{Vergleichstabelle und Bewertung}





\subsection{Konzeptentscheidung für diese Arbeit}





\section{Simulation in COMSOL Multiphysics}


\subsection{Theoretische Grundlagen der Simulation}


\subsection{COMSOL-Modell-Aufbau}


\subsection{Durchgeführte Simulationsstudien}


\subsection{Zusammenfassung Simulations-Ergebnisse}


\section{Test im Freifeld-Raum}


\subsection{Versuchsaufbau}


\subsection{Getestete Exciter-Positionen}


\subsection{Messergebnisse und Auswertung (Messunsicherheit)}


\section{Validierung und Vergleich der Ergebnisse}


\subsection{Vergleich Simulation und Versuch}



\chapter{Umsetzung}

\section{Bau nach Testergebnissen}

\subsection{Materialliste}


\subsection{Finale Messungen}


\section{Subjektive Hör-Validierung}

\subsection{Test-Design (Fragebogen)}


\subsection{Ergebnis Diskussion}


\section{Wirtschaftliche Betrachtung}



\chapter{Fazit und Ausblick}



\section{Zusammenfassung der Ergebnisse}




\bibliography{chapters/bib/litbank1}  			                % Verweis auf Literaturdatenbank *.bib*


\nocite{*} 														%Alle Dokumente anzeigen
\bibliographystyle{natdin}
\setlength{\bibsep}{3mm}                 						% Abstaende im Literaturverzeichnis
\addcontentsline{toc}{chapter}{Literaturverzeichnis}

\end{document}
